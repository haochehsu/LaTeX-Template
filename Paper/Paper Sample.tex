%----------------Set Font Size & Layout----------------%
\documentclass[12pt, letterpaper, oneside]{article}
\usepackage{STYLE, amsfonts}
\hypersetup{colorlinks=true, linkcolor=blue, urlcolor=blue,
citecolor=blue, anchorcolor=blue}
%---------------------Set Margins----------------------%
\usepackage[letterpaper, left=1in, right=1in,
top=1in, bottom=1in, bindingoffset=0mm]{geometry}
%-------------------Header & Footer-------------------%
\fancypagestyle{style1}{
\fancyhf{}
\chead{Title}
%\fancyhead[RO,LE]{Thesis Title}
\fancyfoot[C]{\thepage}
%\renewcommand{\headrulewidth}{0pt}
}
%\pagestyle{style1}
%------------------Title Page Settings------------------%
\title{The Title of the Paper\thanks{I thank Professor Chen and Hao-Che Hsu for useful comments and suggestions. All errors are my own.}}
\author{Your Name\footnote{Department of Economics, University of California-Irvine. E-mail: yourUCInetID@uci.edu. UCI-ID.}}
\date{2020}
%-------------------------Main-------------------------%
\begin{document}
%\setlength{\baselineskip}{16pt}
\doublespacing
\maketitle

\begin{abstract}
    \lipsum[30]
\end{abstract}

\noindent\textbf{Keywords:} Competition, pricing strategy\\

\noindent\textbf{JEL Code:} L10
\newpage

\section{Introduction}
\cite{chen2018} finds a strong network effect. Following the literature on numerically solving the problem \citep{chen2018}, I will simulate the model equilibrium in this paper. \lipsum[31-32]

\subsection{Industry Background}
\lipsum[33]

\section{Data}
I will do a case study and follow the instructions from a website \citep{web-io} to adopt a old but classic model from \cite{art}, an unpublished manuscript. But from another book, \cite{book-tirole} claims that there is bias in this approach. \lipsum[34-35]

\section{Measuring the Competition}
\lipsum[36]

\subsection{Regression Model}
\lipsum[43]

\begin{figure}[h!]
    \centering
    \caption{Here is the figure title.}
    \vspace*{-0.2cm}
    \fbox{\rule[-.5cm]{0cm}{4cm} \rule[-.5cm]{10cm}{0cm}}
    % \includegraphics[scale=0.8]{figure.pdf}}
    \begin{minipage}{0.63\textwidth}
        \vspace{0.15cm}
        {\footnotesize  The figure needs to be \lquote self-contained\rquote. This means that you need to explain the contents of this figure here, in the figure's footnote section regardless whether you have already explained it in the section contents or not.}
    \end{minipage}
\end{figure}

\lipsum[46-47]

\renewcommand*\arraystretch{1.5}%stretch table (height)
\renewcommand{\tabcolsep}{30pt}%set table width
\begin{table}[h!]
    \centering
    \caption{Here is the table title.}
    \vspace*{-0.2cm}
    \fontsize{9.5}{11}\selectfont
    \def\sym#1{\ifmmode^{#1}\else\(^{#1}\)\fi}
    \begin{tabular}{@{}l*{10}{D{.}{.}{7}}@{}}
        \hline\hline
        &\multicolumn{1}{c}{(1)}&\multicolumn{1}{c}{(2)}\\
        &\multicolumn{1}{c}{Calendly users}&\multicolumn{1}{c}{Splitwise Users}\\
        \hline
        Price     &   -2.003\sym{***}&   -3.006\sym{***}\\
        &  (0.357)         &  (0.316)         \\
        Constant    &  199.848\sym{***}&  240.097\sym{***}\\
        & (17.603)         & (15.589)         \\
        \hline
        \(N\)     &       10         &       10         \\
        \hline\hline
        \multicolumn{3}{l}{\footnotesize \hspace{-1.15cm} Standard errors in parentheses. \footnotesize \sym{*} \(p<0.10\), \sym{**} \(p<0.05\), \sym{***} \(p<0.01\).}
    \end{tabular}
    \begin{minipage}{0.705\textwidth}
        \vspace{0.15cm}
        {\footnotesize  The table also needs to be \lquote self-contained\rquote. The explanation of this table is displayed here. This LaTeX code for this table is directly generated by STATA regression results.}
    \end{minipage}
\end{table}

\lipsum[61]

\section{Results}
\lipsum[39]

\section{Conclusion}
\lipsum[40]

\bibliographystyle{JPE}
\bibliography{REFERENCE}

\end{document}